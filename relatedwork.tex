% !TEX root = 15cvpr.tex
\section{Related Work}
Being such a fundamental problem in computer vision community, numerous methods have been proposed for the semantic segmentation task. Here we review the works that most related to ours.

Under the fully supervised settings, Shotton \etal \cite{shotton2006textonboost} formulate semantic segmentation as an conditional random fields (CRF) over image pixels incorporating shape-texture color, location and edge cues in a single unified model. This model is further extended in series papers \cite{kohli2009robust,ladicky2009associative,ladicky2010graph}. For instance, Kohli \etal utilize the higher order potentials \cite{kohli2009robust} as a soft decision to ensure that pixels constituting a particular segment have the same semantic concept, Ladicky \etal extend the higher order potentials to hierarchical structure in \cite{ladicky2009associative} by using multiple segmentations and further integrate label co-occurrence statistics in \cite{ladicky2010graph}.

However, there has been little work in the weakly supervised setting due to the fact that it is more challenging than the fully supervised task. Verbeek and Triggs \cite{verbeek2007region} make the first attempt to learn a semantic segmentation model from image-level tagged data. They leverage several appearance descriptors to learn the latent aspect model via probabilistic Latent Semantic Analysis (pLSA). Furthermore, the spanning tree structure and Markov Fields are integrated with the aspect model in order to take the spatial information into consideration. In \cite{vezhnevets2010towards}, Vezhnevets and Buhmann 