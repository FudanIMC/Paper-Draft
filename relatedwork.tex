% !TEX root = 15cvpr.tex
\section{Related Work}
Being such a fundamental problem in computer vision community, numerous methods have been proposed for the semantic segmentation task in the fully supervised settings. Shotton \etal \cite{shotton2006textonboost} formulate semantic segmentation as an Conditional Random Fields (CRF) over image pixels incorporating shape-texture color, location and edge cues in a single unified model. This model is further extended in series papers \cite{kohli2009robust,ladicky2009associative,ladicky2010graph}. For instance, Kohli \etal utilize the higher order potentials \cite{kohli2009robust} as a soft decision to ensure that pixels constituting a particular segment have the same semantic concept. Ladicky \etal extend the higher order potentials to hierarchical structure in \cite{ladicky2009associative} by using multiple segmentations and further integrate label co-occurrence statistics in \cite{ladicky2010graph}. However, these methods heavily rely on pixel-level annotations during the training stage.

Comparison with fully supervised semantic segmentation, there has been little work in the weakly supervised settings due to the fact that it is more challenging than the fully supervised task. Verbeek and Triggs \cite{verbeek2007region} make the first attempt to learn a semantic segmentation model from image-level tagged data. They leverage several appearance descriptors to learn the latent aspect model via probabilistic Latent Semantic Analysis (pLSA) \cite{hofmann1999probabilistic}. Furthermore, the spanning tree structure and Markov Fields are integrated with the aspect model in order to take the spatial information into consideration. In \cite{vezhnevets2010towards}, Vezhnevets and Buhmann cast the weakly supervised task as a multi-instance multi-task learning problem with the framework of Semantic Texton Forest (STF) \cite{shotton2008semantic}. Based on \cite{vezhnevets2010towards}, Vezhnevets \etal \cite{vezhnevets2011weakly,vezhnevets2012weakly} integrate the latent correlations among the superpixels belong to different images which share the same labels into Conditional Random Field (CRF). Xu \etal \cite{xu2014tell} simplify the previous complicated framework by a graphical model that encodes the presence/absence of a class as well as the assignments of semantic labels to superpixels.

However, all these approaches are based on the assumption that the initial image-level labels are clean and complete. It is not a practical requirement in many real-world applications. Although sharing similarities with both fully supervised and weakly supervised semantic segmentation, an significant and novel difficulty we address in this paper is the issue of noisy annotations (\eg, labels can be incorrect and incomplete), which makes the task more challenging and intractable. To tackle this problem, we investigate label correlations, which are neglected by previous weakly supervised methods \cite{verbeek2007region,vezhnevets2010towards,vezhnevets2011weakly,vezhnevets2012weakly,xu2014tell}, based on not only label co-occurrence statistics but also the visual contextual cues.

Besides, we take topic model generated by an unsupervised method as a mid-level representation of superpixels, while other methods (\eg, \cite{vezhnevets2011weakly,xu2014tell}) only use the appearance model as a low-level representation, to narrow down the gap between semantic space and feature space, in the meantime, to make the whole framework more stable under the noisy condition. Unlike previous weakly supervised methods (\eg, \cite{vezhnevets2012weakly,xu2014tell}), we also utilize multiple scale segmentations trying to avoid the weakness of choice of segmentation which cannot cover all the quantization level of objects.

\if
Being such a fundamental problem in computer vision community, numerous methods have been proposed for the semantic segmentation task. Here we review the works that most related to ours.

And this assumption does not always hold in real-world applications.

Different from these weakly supervised methods, our approach imposes no more extra prerequisites on the initial image-level labels (\eg labels can be incorrect and incomplete), which makes the task more challenging and intractable.

Other works use cluster-based or classifier-based methods. For example, Liu \etal \cite{liu2013weakly} leverage the spectral clustering and discriminative clustering techniques to investigate the relationship between feature space and semantic space, and solve it as an optimization problem within weakly supervised constraints. Zhang \etal \cite{zhang2013sparse} address weakly supervised problem by proposing a classifier evaluation criterion and replacing training stage with evaluating stage to obtain the superpixel-level classifiers.
\fi
