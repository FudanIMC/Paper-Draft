% !TEX root = 15cvpr.tex
\section{Introduction}
Aiming to assign each pixel in an image to one of predefined semantic categories, semantic segmentation is an attractive but challenging task in computer vision community. In the past few years, many different methods \cite{csurka2011efficient,gonfaus2010harmony,ladicky2009associative,nowozin2010parameter,shotton2008semantic,shotton2006textonboost,singh2013nonparametric,verbeek2007scene,yang2007multiple,yao2012describing} have been proposed for this task. Notwithstanding significant improvements they have achieved, most of them rely on full supervision: each pixel of the image for training is manually labeled by humans. Since annotating training images is time-consuming and labor-intensive, only a subset of large-scale image dataset can be labeled, the performance of fully supervised methods is inherently limited so that such system cannot be widely applied in practice.

Recently, a few works have been proposed to address the semantic segmentation problem under the weakly supervised settings, where each training image is annotated by image-level labels specifying which classes are present but no pixel-level annotation is given \cite{verbeek2007region,vezhnevets2010towards,vezhnevets2011weakly,vezhnevets2012weakly,xu2014tell,zhang2013sparse}. Comparing to the trandtional supervised semantic segmentation, such weakly supervised method is more flexible in real-world applications for the image-level annotated images are much easier to obtain. With the prevalence of photo sharing websites and collaborative image tagging system, such as Flickr, which host vast of digital images with user provided tags, this weakly supervised setting for image parsing become feasible.

It is worth noting that the annotations of collaboratively-tagged images may not be accurate (incorrect or incomplete) in practice. However such noisely tagged annotation has been ignored in recent work and  there are some extra prerequisites (\eg labels must be precise and complete), actually, for the inital image-level labels in existed weakly supervised semantic segmentation systems. To be honest, collecting such training images that satisfy all these constraints is still a labor-intensive task. Figure \ref{fig:noisylabel} illustrates a set of representative real-world images and its associated tags. We can observe that only limited tags accurately describe the visual content of the image, while ohter tags are imprecise. Meanwhile, some important tags, which are highly associated with the image, are missing. Therefore, the main challenge lies in how to utilize the noisily labeled images for semantic segmentation. 

In this paper, we present a weakly supervised method which achieves results competitive with fully supervised methods.The mainly contributions of this paper are summarized as follows:
\begin{enumerate}
  \item A new
  \item
  \item A new joint inference with alternate procedure
\end{enumerate}

\if
 Fortunately, owing to the collaborative image tagging system, \eg Flickr, we can easily obtain a large mount of manually labeled images provided by Internet users, though these image-level labels might be noisy (incorrect or incomplete). Therefore, the main challenge lies in how to utilize the noisily labeled images for semantic segmentation (see Figure \ref{fig:noisylabel} for an illustration).


 Moreover, most existing semantic segmentation methods, either fully or weakly supervised, depend on a single choice of image partitioning (quantization). The precise quantization of an image is of significance, and it is less likely to obtain a common optimal quantization (partitioning) level suitable for every object. To overcome this problem, \cite{hoiem2005geometric,kohli2009robust,ladicky2009associative,nowozin2010parameter,russell2006using} used multiple segmentations of the image and achieved good performances by heuristic strategies or enforcing label consistency with higher order potential.
\fi

\begin{figure}[t]
\begin{center}
\fbox{\rule{0pt}{2in} \rule{0.9\linewidth}{0pt}}
   %\includegraphics[width=0.8\linewidth]{egfigure.eps}
\end{center}
   \caption{Example of caption.  It is set in Roman so that mathematics
   (always set in Roman: $B \sin A = A \sin B$) may be included without an
   ugly clash.}
\label{fig:noisylabel}
\end{figure}
